
\documentclass[aspectratio=169,11pt]{beamer}

% --- Beamer setup ---
\usetheme{Madrid}
\setbeamertemplate{navigation symbols}{}
\setbeamertemplate{footline}[frame number]
\usepackage{appendixnumberbeamer}

% --- Language & math ---
\usepackage[T1]{fontenc}
\usepackage[utf8]{inputenc}
\usepackage[french]{babel}
\usepackage{lmodern}
\usepackage{amsmath,amssymb,mathtools}
\usepackage{bm}
\usepackage{graphicx}
\usepackage{xcolor}
\graphicspath{{./}{/mnt/data/}}

% --- Title data ---
\title{Séance 1 — Logique}
\subtitle{Mathématiques pour l’informatique (Bac+2)}
\author{Geovany Batista Polo LAGUERRE}
\institute{FSGA \& Université Quisqueya}
\date{Version du 11/09/2025}

% --- Logos ---
\titlegraphic{%
  \includegraphics[height=1.1cm]{logo.png}\hspace{1.2cm}%
  \includegraphics[height=1.1cm]{logo2.png}%
}
\logo{%
  \includegraphics[height=0.7cm]{logo.png}\hspace{0.4cm}%
  \includegraphics[height=0.7cm]{logo2.png}%
}

% --- Accent color ---
\definecolor{fsgaBlue}{RGB}{0,91,171}
\setbeamercolor{structure}{fg=fsgaBlue}

% --- Section pages ---
\AtBeginSection[]{
  \begin{frame}
    \centering
    \vfill
    {\LARGE \insertsectionhead\par}
    \vfill
  \end{frame}
}

\begin{document}

\begin{frame}
  \titlepage
\end{frame}

% =====================================================================
\begin{frame}{Objectifs de la séance}
\begin{itemize}
  \item Comprendre la notion d'\textbf{assertion} (vrai/faux) et les \textbf{connecteurs logiques}.
  \item Construire et lire des \textbf{tables de vérité}.
  \item Utiliser quelques \textbf{équivalences clés} (De Morgan, contraposée, $p\Rightarrow q \;\Leftrightarrow\; \neg p \vee q$).
  \item Découvrir les \textbf{quantificateurs} $\forall$ et $\exists$ et négations associées.
  \item Pratiquer des \textbf{méthodes de raisonnement} : direct, cas par cas, contraposée, absurde, contre-exemple, aperçu récurrence.
\end{itemize}
\end{frame}

% =====================================================================
\section{Motivation \& bases}

\begin{frame}{Pourquoi la logique ?}
\begin{itemize}
  \item Lever les \textbf{ambiguïtés} du langage naturel ; formaliser conditions et validations.
  \item Langage commun pour \textbf{raisonner}, prouver, \textbf{déboguer} des conditions en code.
  \item Exemples : « fromage \textbf{ou} dessert » (XOR) vs « as \textbf{ou} coeur » (OU inclusif).
\end{itemize}
\end{frame}

\begin{frame}{Assertions (propositions)}
\begin{itemize}
  \item Une \textbf{assertion} est une phrase vraie \emph{ou} fausse (pas les deux).
  \item Ex. « Il pleut. » ; « $2+2=4$ » ; « $\forall x\in\mathbb R,\, x^2\ge 0$ ».
  \item À partir de $P$ et $Q$, on construit de nouvelles assertions via des \textbf{connecteurs}.
\end{itemize}
\end{frame}

% =====================================================================
\section{Connecteurs \& tables de vérité}

\begin{frame}{Connecteurs de base}
\begin{itemize}
  \item \textbf{NON} $\neg P$ : vrai si $P$ est faux.
  \item \textbf{ET} $P\wedge Q$ : vrai si $P$ et $Q$ sont vrais.
  \item \textbf{OU} $P\vee Q$ : vrai si \emph{au moins} l’un des deux est vrai (OU inclusif).
\end{itemize}
\end{frame}

\begin{frame}{Table de vérité — ET}
\centering
\begin{tabular}{c c | c}
$P$ & $Q$ & $P\wedge Q$ \\ \hline
1 & 1 & 1 \\
1 & 0 & 0 \\
0 & 1 & 0 \\
0 & 0 & 0 \\
\end{tabular}
\end{frame}

\begin{frame}{Table de vérité — OU (inclusif)}
\centering
\begin{tabular}{c c | c}
$P$ & $Q$ & $P\vee Q$ \\ \hline
1 & 1 & 1 \\
1 & 0 & 1 \\
0 & 1 & 1 \\
0 & 0 & 0 \\
\end{tabular}
\end{frame}

\begin{frame}{Table de vérité — NON}
\centering
\begin{tabular}{c | c}
$P$ & $\neg P$ \\ \hline
1 & 0 \\
0 & 1 \\
\end{tabular}
\end{frame}

\begin{frame}{OU exclusif (XOR)}
\begin{itemize}
  \item « L’un \emph{ou} l’autre, \emph{pas les deux} ». Utile pour certaines règles métiers.
\end{itemize}
\centering
\begin{tabular}{c c | c}
$P$ & $Q$ & $P \oplus Q$ \\ \hline
1 & 1 & 0 \\
1 & 0 & 1 \\
0 & 1 & 1 \\
0 & 0 & 0 \\
\end{tabular}
\end{frame}

\begin{frame}{Implication $P\Rightarrow Q$}
\begin{itemize}
  \item \textbf{Définition} : $P\Rightarrow Q \;\Leftrightarrow\; \neg P \vee Q$.
  \item \textbf{Fausse seulement} quand $P=1$ et $Q=0$.
  \item Lecture « si $P$ alors $Q$ » ; attention aux idées reçues.
\end{itemize}
\end{frame}

\begin{frame}{Table — Implication et Biconditionnel}
\centering
\begin{tabular}{c c | c c}
$P$ & $Q$ & $P\Rightarrow Q$ & $P\Leftrightarrow Q$ \\ \hline
1 & 1 & 1 & 1 \\
1 & 0 & 0 & 0 \\
0 & 1 & 1 & 0 \\
0 & 0 & 1 & 1 \\
\end{tabular}
\end{frame}

% =====================================================================
\section{Équivalences classiques}

\begin{frame}{Équivalences incontournables}
\begin{itemize}
  \item \textbf{Double négation} : $P \Leftrightarrow \neg\neg P$.
  \item \textbf{De Morgan} : $\neg(P\wedge Q)\Leftrightarrow (\neg P)\vee(\neg Q)$ ; $\neg(P\vee Q)\Leftrightarrow (\neg P)\wedge(\neg Q)$.
  \item \textbf{Distributivité} : $P\wedge(Q\vee R)\Leftrightarrow (P\wedge Q)\vee(P\wedge R)$ ; etc.
  \item \textbf{Contraposée} : $(P\Rightarrow Q)\Leftrightarrow (\neg Q \Rightarrow \neg P)$.
\end{itemize}
\end{frame}

\begin{frame}{Atelier — vérifier des équivalences}
\begin{itemize}
  \item Montrer $(P\Rightarrow Q)\Leftrightarrow (\neg P \vee Q)$ par table.
  \item Tester les deux lois de De Morgan.
  \item Écrire deux variantes logiquement équivalentes d’une même condition (plus lisible).
\end{itemize}
\end{frame}

% =====================================================================
\section{Quantificateurs}

\begin{frame}{Quantificateurs $\forall$ et $\exists$}
\begin{itemize}
  \item $\forall x\in E\, P(x)$ : vrai si $P(x)$ est vrai pour \textbf{tous} les $x$ de $E$.
  \item $\exists x\in E\, P(x)$ : vrai s’il \textbf{existe} au moins un $x$ dans $E$ tel que $P(x)$.
  \item \textbf{Négations} : $\neg(\forall x\, P)\Leftrightarrow \exists x\, \neg P$ ; $\neg(\exists x\, P)\Leftrightarrow \forall x\, \neg P$.
  \item \textbf{Attention à l’ordre} : $\forall x\, \exists y\, P(x,y)$ $\ne$ $\exists y\, \forall x\, P(x,y)$.
\end{itemize}
\end{frame}

\begin{frame}{Exercices — traductions et négations}
\begin{itemize}
  \item Traduire : « Tout réel a un carré $\ge 0$ » ; « Il existe un entier pair $>1000$ ».
  \item Négations : « Tous les étudiants sont à l’heure » ; « Il existe un graphe connexe ... ».
  \item Comparer $\forall x\, \exists y\, P(x,y)$ et $\exists y\, \forall x\, P(x,y)$ sur un exemple concret.
\end{itemize}
\end{frame}

% =====================================================================
\section{Méthodes de raisonnement}

\begin{frame}{Panorama des méthodes}
\begin{itemize}
  \item \textbf{Direct} : enchaîner des implications vraies $P \Rightarrow \cdots \Rightarrow Q$.
  \item \textbf{Cas par cas} : partitionner l’univers (ex. $x\ge 1$ / $x<1$).
  \item \textbf{Contraposée} : prouver $\neg Q \Rightarrow \neg P$.
  \item \textbf{Par l’absurde} : supposer $P$ et $\neg Q$ et conclure à une contradiction.
  \item \textbf{Contre-exemple} : réfuter un universel $\forall x\, P(x)$.
  \item \textbf{Récurrence} : initialisation + hérédité $\Rightarrow$ conclusion.
\end{itemize}
\end{frame}

\begin{frame}{Exemples rapides I}
\begin{itemize}
  \item \textbf{Direct} : si $a,b\in\mathbb Q$ alors $a+b\in\mathbb Q$ (écrire $a=p/q$, $b=p'/q'$).
  \item \textbf{Cas par cas} : $|x-1|\le x^2-x+1$ selon $x\ge 1$ ou $x< 1$.
\end{itemize}
\end{frame}

\begin{frame}{Exemples rapides II}
\begin{itemize}
  \item \textbf{Contraposée} : si $n^2$ est pair alors $n$ est pair $\;\Leftrightarrow\;$ si $n^2$ est impair alors $n$ est impair.
  \item \textbf{Absurde} : $a/(1+b)=b/(1+a)$ avec $a,b>0 \Rightarrow a=b$ (sinon contradiction).
  \item \textbf{Contre-exemple} : pour « $\forall x\, P(x)$ », exhiber un $x$ qui viole $P$.
\end{itemize}
\end{frame}

\begin{frame}{Principe de récurrence (aperçu)}
\begin{itemize}
  \item \textbf{Initialisation} : montrer $P(n_0)$ vrai.
  \item \textbf{Hérédité} : $P(n)\Rightarrow P(n+1)$.
  \item \textbf{Conclusion} : tous les $n\ge n_0$ vérifient $P(n)$.
  \item Exemples : $2^n>n$ ($n\ge 0$) ; $\sum_{k=1}^n k = \frac{n(n+1)}{2}$.
\end{itemize}
\end{frame}

% =====================================================================
\section{Mini-quiz \& synthèse}

\begin{frame}{Mini-quiz (rapide)}
\begin{enumerate}
  \item V/F — $P\Rightarrow Q \;\Leftrightarrow\; \neg P \vee Q$.
  \item Contraposée de « si $p$ alors $q$ » ?
  \item Simplifier $\neg(\neg p \vee q)$.
  \item Donner un exemple où XOR est pertinent.
\end{enumerate}
\end{frame}

\begin{frame}{Wrap-up}
\begin{itemize}
  \item Connecteurs, tables, équivalences (De Morgan, contraposée, $\neg P \vee Q$) 
  \item Quantificateurs \& négations 
  \item Méthodes de raisonnement (direct, cas, contraposée, absurde, contre-ex.) 
  \item \textbf{Suite} : Ensembles, relations \& fonctions ; puis combinatoire / probas discrètes.
\end{itemize}
\end{frame}

\end{document}
