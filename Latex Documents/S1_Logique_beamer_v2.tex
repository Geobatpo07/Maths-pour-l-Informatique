\documentclass[aspectratio=169]{beamer}

% ---------- Theme & language ----------
\usetheme{Madrid}
\usepackage[french]{babel}
\usepackage[T1]{fontenc}
\usepackage[utf8]{inputenc}
\usefonttheme{professionalfonts}
\usepackage{lmodern}
\usepackage{amsmath,amssymb,mathtools}
\usepackage{bm}
\usepackage{xcolor,graphicx}
\usepackage{tikz}
\usepackage{booktabs}

% --- Map common Unicode punctuation for pdfLaTeX ---
\usepackage{textcomp}
%\DeclareUnicodeCharacter{2019}{\textquotesingle} % '
%\DeclareUnicodeCharacter{2013}{--}               % –
%\DeclareUnicodeCharacter{2014}{\textemdash}      % -

% ---------- Colors / style ----------
\definecolor{fsgaBlue}{RGB}{0,91,171}
\setbeamercolor{structure}{fg=fsgaBlue}
\setbeamercolor{title}{fg=white,bg=fsgaBlue}
\setbeamercolor{frametitle}{fg=fsgaBlue}
\setbeamercolor{block title}{bg=fsgaBlue!10,fg=fsgaBlue}
\setbeamercolor{block body}{bg=fsgaBlue!3}
\setbeamertemplate{navigation symbols}{}
\setbeamertemplate{footline}[frame number]

% ---------- Shortcuts ----------
\newcommand{\N}{\mathbb{N}}
\newcommand{\Z}{\mathbb{Z}}
\newcommand{\Q}{\mathbb{Q}}
\newcommand{\R}{\mathbb{R}}
\newcommand{\C}{\mathbb{C}}

% ---------- Meta ----------
\title[Logique - Séance riche]{Logique \& méthodes de raisonnement}
\subtitle{Mathématiques pour Informaticiens}
\author[Geovany B. P. Laguerre]{Geovany Batista Polo LAGUERRE \textbar{} Data Scientist}
\institute[FSGA \& UniQ]{FSGA - Université Quisqueya}
\date{\today}

% Place logo.png / logo2.png next to the .tex if you want them on the title
\titlegraphic{%
  \vspace{-0.8cm}\hspace{0.5cm}%
  \includegraphics[height=1.1cm]{logo.png}\hfill
  \includegraphics[height=1.1cm]{logo2.png}\hspace{0.5cm}
}

% ---------- Theorem-like ----------
\newtheorem{theoreme}{Théorème}
\newtheorem{proposition}{Proposition}
\newtheorem{corollaire}{Corollaire}
\theoremstyle{definition}
\newtheorem{definitionenv}{Définition}
\newtheorem{exemple}{Exemple}
\newtheorem{exercice}{Exercice}

\begin{document}

% ===== Title =====
\begin{frame}
  \titlepage
\end{frame}

\AtBeginSection[]{
  \begin{frame}{Plan}
    \tableofcontents[currentsection]
  \end{frame}
}

% ===== Objectifs =====
\section{Objectifs}
\begin{frame}{Objectifs de la séance}
\begin{itemize}
  \setlength\itemsep{3pt}
  \item Lever les ambiguïtés du langage naturel par un langage formel ;
  \item Manipuler connecteurs et tables de vérité ;
  \item Comprendre implication et équivalence ; pratiquer les quantificateurs ;
  \item Savoir utiliser : raisonnement direct, cas, contraposée, absurde, récurrence ;
  \item Ancrer par des mini-exercices corrigés.
\end{itemize}
\end{frame}

% ===== Pourquoi la logique ? Ambiguïtés & rigueur =====
\section{Pourquoi la logique ?}
\begin{frame}{Pourquoi la logique ? Ambiguïtés \& rigueur}
\begin{block}{Ambiguïté typique : « ou »}
\begin{itemize}
  \setlength\itemsep{2pt}
  \item \textbf{Ou inclusif} : $P \lor Q$ (vrai si au moins l'un est vrai) ;
  \item \textbf{Ou exclusif (XOR)} : $P \oplus Q$ (vrai si exactement un des deux est vrai).
\end{itemize}
\end{block}
\begin{block}{But du cours}
Passer d'énoncés informels à des \textbf{assertions} formelles, puis à des \textbf{preuves} courtes et rigoureuses.
\end{block}
\end{frame}

% ===== Assertions =====
\section{Assertions}
\begin{frame}{Assertions (propositions)}
\begin{definitionenv}
Une \textbf{assertion} est une phrase qui a une valeur de vérité (vraie ou fausse, pas les deux).
\end{definitionenv}
Exemples : « Il pleut », « $2+2=4$ », « $\forall x\in\R,\ x^2\ge 0$ ».
\end{frame}

% ===== Connecteurs =====
\section{Connecteurs}
\begin{frame}{Connecteurs logiques - définitions}
\begin{columns}[T]
\column{0.52\textwidth}
\begin{itemize}
  \setlength\itemsep{2pt}
  \item NON : $\neg P$ ;
  \item ET : $P\land Q$ ;
  \item OU : $P\lor Q$ ;
  \item Implication : $P\Rightarrow Q \equiv \neg P \lor Q$ ;
  \item Équivalence : $P\Leftrightarrow Q \equiv (P\Rightarrow Q)\land(Q\Rightarrow P)$.
\end{itemize}
\column{0.48\textwidth}
\begin{block}{Équivalences utiles}
\begin{itemize}
  \setlength\itemsep{2pt}
  \item $\neg(P\land Q) \equiv (\neg P)\lor(\neg Q)$ ;
  \item $\neg(P\lor Q) \equiv (\neg P)\land(\neg Q)$ ;
  \item $(P\land(Q\lor R)) \equiv (P\land Q)\lor(P\land R)$ ;
  \item $(P\Rightarrow Q) \equiv (\neg Q \Rightarrow \neg P)$ (contraposée).
\end{itemize}
\end{block}
\end{columns}
\end{frame}

% ===== Tables de vérité =====
\section{Tables de vérité}
\begin{frame}{Tables de vérité - ET, OU, NON}
\begin{columns}[T]
\column{0.33\textwidth}
\begin{tabular}{c c c}
\toprule
$P$ & $Q$ & $P\land Q$\\\midrule
1&1&1\\
1&0&0\\
0&1&0\\
0&0&0\\\bottomrule
\end{tabular}
\column{0.33\textwidth}
\begin{tabular}{c c c}
\toprule
$P$ & $Q$ & $P\lor Q$\\\midrule
1&1&1\\
1&0&1\\
0&1&1\\
0&0&0\\\bottomrule
\end{tabular}
\column{0.33\textwidth}
\begin{tabular}{c c}
\toprule
$P$ & $\neg P$\\\midrule
1&0\\
0&1\\\bottomrule
\end{tabular}
\end{columns}
\end{frame}

% ===== XOR =====
\section{XOR}
\begin{frame}{OU exclusif (XOR)}
\begin{tabular}{c c c}
\toprule
$P$ & $Q$ & $P\oplus Q$\\\midrule
1&1&0\\
1&0&1\\
0&1&1\\
0&0&0\\\bottomrule
\end{tabular}
\end{frame}

% ===== Implication =====
\section{Implication}
\begin{frame}{Implication (P ⇒ Q)}
\begin{itemize}
  \setlength\itemsep{3pt}
  \item $P\Rightarrow Q \equiv \neg P \lor Q$ ;
  \item Fausse seulement si $P$ vraie et $Q$ fausse ;
  \item Contraposée : $(\neg Q \Rightarrow \neg P)$.
\end{itemize}
\end{frame}

% ===== Biconditionnel =====
\section{Biconditionnel}
\begin{frame}{Biconditionnel (P ⇔ Q)}
\begin{tabular}{c c c c}
\toprule
$P$ & $Q$ & $P\Rightarrow Q$ & $P\Leftrightarrow Q$\\\midrule
1&1&1&1\\
1&0&0&0\\
0&1&1&0\\
0&0&1&1\\\bottomrule
\end{tabular}
\end{frame}

% ===== Quantificateurs =====
\section{Quantificateurs}
\begin{frame}{Quantificateurs - \(\forall\) et \(\exists\)}
\begin{itemize}
  \setlength\itemsep{3pt}
  \item $\forall x\in E\, P(x)$ : vrai si $P$ vaut pour tout $x$ ;
  \item $\exists x\in E\, P(x)$ : vrai si $P$ vaut pour au moins un $x$ ;
  \item Négations : $\neg(\forall x\,P)\equiv \exists x\,\neg P$ ; $\neg(\exists x\,P)\equiv \forall x\,\neg P$ ;
  \item Attention à l'ordre : $\forall x\exists y \neq \exists y\forall x$ en général.
\end{itemize}
\end{frame}

% ===== Méthodes de raisonnement =====
\section{Méthodes de raisonnement}
\begin{frame}{Méthodes de raisonnement - panorama}
\begin{itemize}
  \setlength\itemsep{3pt}
  \item Direct ; par cas ; par contraposée ; par l'absurde ;
  \item Par récurrence (simple/forte) ; bon ordre / plus petit contre-exemple ;
  \item Invariants (jeux/processus).
\end{itemize}
\end{frame}

% ===== Exemples de preuves courtes =====
\section{Exemples de preuves courtes}
\begin{frame}{Exemples de preuves courtes}
\begin{block}{Direct - somme de rationnels}
Si $a,b\in\Q$, écrire $a=\frac{p}{q}$, $b=\frac{r}{s}$, alors $a+b=\frac{ps+rq}{qs}\in\Q$.
\end{block}
\begin{block}{Cas par cas - \(|x-1|\le x^2-x+1\)}
Étudier $x\ge 1$ (alors $|x-1|=x-1$) et $x<1$ ($|x-1|=1-x$) : dans les deux cas l'inégalité se vérifie.
\end{block}
\end{frame}

\begin{frame}{Contraposée \& Absurde}
\begin{block}{Contraposée - si \(n^2\) pair alors \(n\) pair}
Supposer $n$ impair $\Rightarrow n=2k+1 \Rightarrow n^2=4k(k+1)+1$ impair. Donc contraposée prouvée.
\end{block}
\begin{block}{Absurde - \(\frac{a}{1+b}=\frac{b}{1+a}\Rightarrow a=b\) (a,b>0)}
Supposer $a\ne b$. Par croisement : $a(1+a)=b(1+b)\Rightarrow a-a^2=b-b^2\Rightarrow (a-b)(1-a-b)=0$.  
Or $a\ne b$ et $a,b>0\Rightarrow 1-a-b<0$ impossible avec égalité. Contradiction $\Rightarrow a=b$.
\end{block}
\end{frame}

% ===== Contre-exemple / Récurrence =====
\section{Contre-exemple / Récurrence}
\begin{frame}{Contre-exemple \& Récurrence}
\begin{block}{Contre-exemple}
Tous les entiers >0 sont somme de deux carrés est faux : $3$ ne convient pas (et $7$ non plus).
\end{block}
\begin{block}{Récurrence - exemple}
Montrer $\sum_{k=1}^n k = \frac{n(n+1)}{2}$ : initialisation $n=1$, hérédité $n\rightarrow n+1$ par ajout de $(n+1)$.
\end{block}
\end{frame}

% ===== Mini-exercices =====
\section{Mini-exercices}
\begin{frame}{Mini-quiz (rapide)}
\begin{enumerate}
  \setlength\itemsep{3pt}
  \item V/F : $P\Rightarrow Q \equiv \neg P \lor Q$ ;
  \item Donner la table de vérité de XOR ;
  \item Écrire la négation de $\forall x\in\R,\ \exists y\in\R,\ y>x$.
\end{enumerate}
\end{frame}

% ===== Wrap-up =====
\section{Wrap-up}
\begin{frame}{Wrap-up \& suite}
\begin{itemize}
  \setlength\itemsep{3pt}
  \item Outils : connecteurs, équivalences, quantificateurs ;
  \item Méthodes : direct, cas, contraposée, absurde, récurrence, invariants ;
  \item Prochaine étape : mise en pratique sur ensembles/relations/fonctions.
\end{itemize}
\end{frame}

\end{document}
