\documentclass[11pt,a4paper]{article}

% --- Encodage, langue, math ---
\usepackage[T1]{fontenc}
\usepackage[utf8]{inputenc}
\usepackage[french]{babel}
\usepackage{lmodern}
\usepackage{amsmath,amssymb}
\usepackage{enumitem} % pour [label=\arabic*.] dans enumerate
\usepackage{microtype} % meilleures césures/justifications
\usepackage{hyperref}

% --- Ensembles usuels & opérateurs logiques ---
\newcommand{\R}{\mathbb{R}}
\newcommand{\Q}{\mathbb{Q}}
\newcommand{\N}{\mathbb{N}}
\newcommand{\impl}{\Rightarrow}
\newcommand{\equi}{\Leftrightarrow}

% --- Mise en page & logos ---
\usepackage[a4paper,margin=2.2cm]{geometry}
\usepackage{graphicx}
\graphicspath{{./}} % Assure-toi d'avoir logo.png et logo2.png ici
\usepackage{xcolor}
\usepackage{array} % pour colonnes p{...} dans tabular

% --- Couleur d'accent (FSGA) ---
\definecolor{fsgaBlue}{RGB}{0,91,171}

% --- Méta (modifie ces 6 lignes) ---
\newcommand{\Titre}{TD - Logique \& Raisonnement}
\newcommand{\SousTitre}{Mathématiques pour Informaticiens}
\newcommand{\Enseignant}{Geovany Batista Polo LAGUERRE \textbar{} Data Scientist}
\newcommand{\Institution}{FSGA \& Université Quisqueya}
\newcommand{\Semestre}{Semestre 1}
\newcommand{\Annee}{2025--2026}

\begin{document}
\begin{titlepage}
  \centering

  % Ligne logos
  \makebox[\textwidth]{%
    \includegraphics[height=1.6cm]{logo.png}\hfill
    \includegraphics[height=1.6cm]{logo2.png}%
  }

  \vspace{1.8cm}

  % Titre
  {\LARGE\bfseries \Titre\par}
  \vspace{0.35cm}
  {\large\itshape \SousTitre\par}

  \vspace{1.2cm}
  % Barre d'accent
  {\color{fsgaBlue}\rule{\textwidth}{1.4pt}}

  \vspace{1.2cm}
  % Informations (2e colonne large pour éviter les débordements)
  \begin{tabular}{@{}l p{0.72\textwidth}}
    \textbf{Enseignant :} & \Enseignant \\
    \textbf{Institution :} & \Institution \\
    \textbf{Semestre :} & \Semestre \\
    \textbf{Année académique :} & \Annee \\
    \textbf{Version :} & \today \\
  \end{tabular}

  \vfill
  % Note de bas de page (facultatif)
  {\small Ce document de Travaux Dirigés regroupe des exercices de logique (propositions, tables de vérité, équivalences, quantificateurs, méthodes de preuve).}

\end{titlepage}

% --- Table des matières en dehors de la page de garde ---
\tableofcontents
\clearpage

% ===================== INTRO =====================
\section*{Quelques motivations}
\addcontentsline{toc}{section}{Quelques motivations}
Les mathématiques fournissent un \emph{langage rigoureux} pour lever les ambiguïtés du langage naturel et une méthode pour décider du \emph{vrai} et du \emph{faux}.
Exemples classiques : le mot « ou » peut être \emph{exclusif} (restaurant « fromage ou dessert ») ou \emph{inclusif} (jeu de cartes « as ou c\oe{}urs »).
Les notions complexes (ex. continuité) se formalisent mieux avec des quantificateurs :
\[
\forall \varepsilon>0\, \exists \delta>0\, \forall x \in I \ (|x-x_0|<\delta \impl |f(x)-f(x_0)|<\varepsilon).
\]

\section{Logique}
\subsection{Assertions}
Une \textbf{assertion} est une phrase \emph{vraie ou fausse}, mais pas les deux. Exemples : « Il pleut. »,
« $2+2=4$ », « $\forall x\in\R,\ x^2\ge 0$ ». À partir de deux assertions $P$ et $Q$, on fabrique de nouvelles assertions à l'aide de \textbf{connecteurs}.

\subsection{Connecteurs et tables de vérité}
\paragraph{NON, ET, OU (inclusif).}
\begin{center}
\begin{tabular}{>{\centering\arraybackslash}m{1.2cm} >{\centering\arraybackslash}m{1.2cm} >{\centering\arraybackslash}m{2.2cm}}
\multicolumn{3}{c}{\textbf{ET} $P\wedge Q$}\\\hline
$P$ & $Q$ & $P\wedge Q$\\\hline
1 & 1 & 1\\
1 & 0 & 0\\
0 & 1 & 0\\
0 & 0 & 0\\\hline
\end{tabular}\hspace{0.6cm}
\begin{tabular}{>{\centering\arraybackslash}m{1.2cm} >{\centering\arraybackslash}m{1.2cm} >{\centering\arraybackslash}m{2.2cm}}
\multicolumn{3}{c}{\textbf{OU} $P\vee Q$}\\\hline
$P$ & $Q$ & $P\vee Q$\\\hline
1 & 1 & 1\\
1 & 0 & 1\\
0 & 1 & 1\\
0 & 0 & 0\\\hline
\end{tabular}\hspace{0.6cm}
\begin{tabular}{>{\centering\arraybackslash}m{1.2cm} >{\centering\arraybackslash}m{2.2cm}}
\multicolumn{2}{c}{\textbf{NON} $\neg P$}\\\hline
$P$ & $\neg P$\\\hline
1 & 0\\
0 & 1\\\hline
\end{tabular}
\end{center}

\paragraph{Implication et biconditionnel.}
Par définition, $P\impl Q \ \equi\ \neg P \vee Q$ ; elle est \emph{fausse seulement} pour $(P,Q)=(1,0)$.
Le \textbf{biconditionnel} est $P\equi Q \ \equi\ (P\impl Q)\wedge(Q\impl P)$.
\begin{center}
\begin{tabular}{>{\centering\arraybackslash}m{1.2cm} >{\centering\arraybackslash}m{1.2cm} >{\centering\arraybackslash}m{2.2cm} >{\centering\arraybackslash}m{2.2cm}}
$P$ & $Q$ & $P\impl Q$ & $P\equi Q$\\\hline
1 & 1 & 1 & 1\\
1 & 0 & 0 & 0\\
0 & 1 & 1 & 0\\
0 & 0 & 1 & 1\\\hline
\end{tabular}
\end{center}

\paragraph{Équivalences classiques.}
\begin{itemize}
  \item Double négation : $P \equi \neg\neg P$.
  \item De Morgan : $\neg(P\wedge Q)\equi (\neg P)\vee(\neg Q)$ ; $\neg(P\vee Q)\equi (\neg P)\wedge(\neg Q)$.
  \item Distributivité : $P\wedge(Q\vee R)\equi (P\wedge Q)\vee(P\wedge R)$ et $P\vee(Q\wedge R)\equi (P\vee Q)\wedge(P\vee R)$.
  \item Contraposée : $(P\impl Q)\equi (\neg Q\impl \neg P)$.
\end{itemize}

\subsection{Quantificateurs}
Pour une propriété $P(x)$ et un ensemble $E$ :
\begin{itemize}
  \item $\forall x\in E\ P(x)$ : vrai si $P(x)$ est vraie pour \emph{tous} les $x\in E$ ;
  \item $\exists x\in E\ P(x)$ : vrai s'il existe \emph{au moins un} $x\in E$ tel que $P(x)$.
\end{itemize}
\textbf{Négations :} $\neg(\forall x\, P)\equi \exists x\, \neg P$ ; \ $\neg(\exists x\, P)\equi \forall x\, \neg P$.\\
\textbf{Attention à l'ordre :} en général $\forall x\, \exists y\, P(x,y)\ne \exists y\, \forall x\, P(x,y)$.

\section{Raisonnements}
\subsection{Raisonnement direct}
Pour montrer $P\impl Q$, on suppose $P$ vraie et on déduit $Q$ par calculs/arguments.
\paragraph{Exemple.} Si $a,b\in\Q$ alors $a+b\in\Q$ (écrire $a=\frac{p}{q}$, $b=\frac{p'}{q'}$ et sommer).

\subsection{Cas par cas}
On partitionne les possibilités et on traite chaque cas.
\paragraph{Exemple.} Pour $x\in\R$, $|x-1|\le x^2-x+1$ selon $x\ge 1$ ou $x<1$.

\subsection{Contraposée}
Utiliser $(P\impl Q)\equi (\neg Q\impl \neg P)$.
\paragraph{Exemple.} Si $n^2$ est pair alors $n$ est pair $\ \equi\ $ si $n$ est impair alors $n^2$ est impair.

\subsection{Absurde}
Supposer $P$ et $\neg Q$ et obtenir une contradiction.
\paragraph{Exemple.} Si $\dfrac{a}{1+b}=\dfrac{b}{1+a}$ avec $a,b>0$ alors $a=b$.

\subsection{Contre-exemple}
Pour réfuter $\forall x\in E\, P(x)$, exhiber $x\in E$ tel que $\neg P(x)$.
\paragraph{Exemple.} « Tout entier positif est somme de trois carrés » est fausse : $7$ ne convient pas.

\subsection{Récurrence (aperçu)}
Pour $P(n)$, $n\in\N$ : \textbf{initialisation} $P(n_0)$ ; \textbf{hérédité} $P(n)\impl P(n+1)$ ; \textbf{conclusion}.
\paragraph{Exemples.} $2^n>n$ pour $n\ge 0$ ; $\sum_{k=1}^n k=\frac{n(n+1)}{2}$.

\section*{Mini-exercices}
\addcontentsline{toc}{section}{Mini-exercices}
\begin{enumerate}
  \item Écrire la table de vérité du \textbf{ou exclusif} (XOR).
  \item Vérifier $\neg(P\wedge Q)\equi (\neg P)\vee(\neg Q)$ par la table.
  \item Écrire la négation de « $P\impl Q$ ».
  \item Rédiger les démonstrations des équivalences ci-dessus.
  \item Donner la négation de « $P\wedge(Q\vee R)$ ».
  \item Traduire avec quantificateurs : « Tout réel a un carré positif » et écrire la négation.
  \item Récurrence : montrer $\sum_{k=1}^n k=\frac{n(n+1)}{2}$ ; montrer $(1+x)^n>1+nx$ pour $x>0$.
\end{enumerate}

\vfill
\noindent\textit{Auteurs du chapitre d'origine : Arnaud Bodin, Benjamin Boutin, Pascal Romon.}

\end{document}
