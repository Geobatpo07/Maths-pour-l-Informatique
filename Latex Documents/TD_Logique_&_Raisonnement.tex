\documentclass[11pt,a4paper]{article}

% --- Encodage, langue, math ---
\usepackage[T1]{fontenc}
\usepackage[utf8]{inputenc}
\usepackage[french]{babel}
\usepackage{lmodern}
\usepackage{amsmath,amssymb}
\usepackage{enumitem} % pour [label=\arabic*.] dans enumerate
\usepackage{microtype} % meilleures césures/justifications

% --- Ensembles usuels ---
\newcommand{\R}{\mathbb{R}}
\newcommand{\Q}{\mathbb{Q}}

% --- Mise en page & logos ---
\usepackage[a4paper,margin=2.2cm]{geometry}
\usepackage{graphicx}
\graphicspath{{./}} % Assure-toi d'avoir logo.png et logo2.png ici
\usepackage{xcolor}
\usepackage{array} % pour colonnes p{...} dans tabular

% --- Couleur d'accent (FSGA) ---
\definecolor{fsgaBlue}{RGB}{0,91,171}

% --- Méta (modifie ces 6 lignes) ---
\newcommand{\Titre}{TD - Logique \& Raisonnement}
\newcommand{\SousTitre}{Mathématiques pour Informaticiens}
\newcommand{\Enseignant}{Geovany Batista Polo LAGUERRE \textbar{} Data Scientist}
\newcommand{\Institution}{FSGA - Université Quisqueya}
\newcommand{\Semestre}{Semestre 1}
\newcommand{\Annee}{2025-2026}

\begin{document}
\begin{titlepage}
  \centering

  % Ligne logos
  \makebox[\textwidth]{%
    \includegraphics[height=1.6cm]{logo.png}\hfill
    \includegraphics[height=1.6cm]{logo2.png}%
  }

  \vspace{1.8cm}

  % Titre
  {\LARGE\bfseries \Titre\par}
  \vspace{0.35cm}
  {\large\itshape \SousTitre\par}

  \vspace{1.2cm}
  % Barre d'accent
  {\color{fsgaBlue}\rule{\textwidth}{1.4pt}}

  \vspace{1.2cm}
  % Informations (2e colonne large pour éviter les débordements)
  \begin{tabular}{@{}l p{0.72\textwidth}}
    \textbf{Enseignant :} & \Enseignant \\
    \textbf{Institution :} & \Institution \\
    \textbf{Semestre :} & \Semestre \\
    \textbf{Année académique :} & \Annee \\
    \textbf{Version :} & \today \\
  \end{tabular}

  \vfill
  % Note de bas de page (facultatif)
  {\small Ce document de Travaux Dirigés regroupe des exercices de logique (propositions, tables de vérité, équivalences, quantificateurs, méthodes de preuve).}

\end{titlepage}

\newpage

\section*{Exercice 1 - Écrire la négation des propositions suivantes}
\begin{enumerate}[label=\arabic*.]
  \item \emph{Toutes les voitures rapides sont rouges.}
  \item \emph{Il existe un mouton écossais dont au moins un côté est noir.}
  \item \emph{Pour tout $\varepsilon>0$, il existe $q\in\Q_{>0}$ tel que $0<q<\varepsilon$.}
  \item \emph{Pour tout $x\in\R$, on a $x^{2}<0$.}
\end{enumerate}

\section*{Exercice 2 - Énoncer la négation des assertions suivantes}
\begin{enumerate}[label=\arabic*.]
  \item \emph{Tout triangle rectangle possède un angle droit.}
  \item \emph{Dans toutes les prisons, tous les détenus détestent tous les gardiens.}
  \item \emph{Pour tout entier $x$ il existe un entier $y$ tel que, pour tout entier $z$, la relation $z<y$ implique la relation $z<x+1$.}
\end{enumerate}

\section*{Exercice 3 - Propositions et négations}
Soient $P,Q,R$ des propositions. Dans chacun des cas suivants, les propositions citées sont-elles la négation l'une de l'autre ?
\begin{enumerate}[label=\arabic*.]
  \item $(P \land Q)$ ;\quad $(\neg P \land \neg Q)$
  \item $(P \Rightarrow Q)$ ;\quad $(\neg Q \Rightarrow \neg P)$
  \item $(P \lor Q)$ ;\quad $(P \land Q)$
\end{enumerate}

\section*{Exercice 4 - Écrire la négation des propositions suivantes}
Soient $a,b,c\in\R$.
\begin{enumerate}[label=\arabic*.]
  \item $a \le -2 \;\text{ou}\; a \ge 3$.
  \item $a \le 5 \;\text{et}\; a>-1$.
  \item $a \le 5 \;\text{ou}\; 3>c$.
\end{enumerate}

\section*{Exercice 5 - Traduction en langage propositionnel}
On suppose vraies les propositions suivantes : \quad
$p$ :  les chiens aboient  \quad et \quad $q$ :  la caravane passe .
\begin{enumerate}[label=\alph*)]
  \item Si la caravane passe, alors les chiens aboient.
  \item Les chiens n'aboient pas.
  \item La caravane ne passe pas ou les chiens aboient.
  \item Les chiens n'aboient pas et la caravane ne passe pas.
\end{enumerate}

\section*{Exercice 6 - Équivalence entre A et B ? Donner l'implication vraie s'il y en a une.}
\paragraph{Exemple 1.}
\textbf{Proposition A} : Pour toute porte, il existe une clé qui ouvre la porte.\\
\textbf{Proposition B} : Il existe une clé telle que, pour toute porte, cette clé ouvre la porte.

\medskip
\paragraph{Exemple 2.}
\textbf{Proposition A} : Pour tout $x\in\mathbb{R}$, il existe $y\in\mathbb{R}$ tel que $y<x$.\\
\textbf{Proposition B} : Il existe $y\in\mathbb{R}$ tel que, pour tout $x\in\mathbb{R}$, $y<x$.

\section*{Exercice 7 - Relations logiques entre assertions}
Soit l'univers  les hommes . Examiner les relations logiques (compatibilité, contradiction, contraires,
subalternes, etc.) entre :
\begin{enumerate}[label=\Alph*)]
  \item Tous les hommes sont mortels.
  \item Tous les hommes sont immortels.
  \item Aucun homme n'est mortel.
  \item Aucun homme n'est immortel.
  \item Il existe des hommes immortels.
  \item Il existe des hommes mortels.
\end{enumerate}

\section*{Exercice 8 - Ou exclusif (XOR)}
On dit que  $P$ ou exclusif $Q$  est vrai si $P$ ou $Q$ est vrai, mais pas simultanément $P$ et $Q$.\\
\textit{Écrire la table de vérité de $P\oplus Q$.}
\begin{center}
\renewcommand{\arraystretch}{1.15}
\begin{tabular}{c c | c}
$P$ & $Q$ & $P\oplus Q$ \\ \hline
  &   &   \\
  &   &   \\
  &   &   \\
  &   &   \\
\end{tabular}
\end{center}

\section*{Exercice 9 - Traduction en langage naturel}
En interprétant $p$ par  je pars , $q$ par  tu restes , et $r$ par  il n'y a personne ,
traduire chacune des formules suivantes en phrases du langage courant :
\begin{enumerate}[label=\arabic*.]
  \item $(p \land \neg q)\Rightarrow r$
  \item $(\neg p \lor \neg q)\Rightarrow \neg r$
\end{enumerate}

\section*{Exercice 10 - Évaluer les formules (en ne tenant compte que des valeurs données)}
Évaluer les expressions suivantes en utilisant uniquement les valeurs indiquées (les autres variables restent indéterminées).
\begin{enumerate}[label=\arabic*)]
  \item $q \Rightarrow (p \Rightarrow r)$, \quad avec $q=\mathrm{0}$.
  \item $p \land (p \lor q)$, \quad avec $q=\mathrm{1}$.
  \item $p \lor (q \Rightarrow r)$, \quad avec $q=\mathrm{0}$.
\end{enumerate}
% Indication : utiliser que $(\mathrm{F}\Rightarrow X)=\mathrm{V}$ et que $(\mathrm{F}\Rightarrow r)=\mathrm{V}$.

\section*{Exercice 11 - Démontrer par tables de vérité}
Montrer, à l'aide des tables de vérité, que les tautologies suivantes sont vraies :
\[
\bigl((P\Rightarrow Q)\land(Q\Rightarrow R)\bigr)\Rightarrow(P\Rightarrow R),
\qquad
\neg(P\lor Q)\ \Leftrightarrow\ (\neg P\land \neg Q),
\qquad
\neg(P\land Q)\ \Leftrightarrow\ (\neg P\lor \neg Q).
\]
% (Vous pouvez construire la table sur 8 lignes pour P,Q,R dans la première formule.)

\section*{Exercice 12 - Tables de vérité \& classification}
Évaluer les formules suivantes par tables de vérité, puis indiquer pour chacune si elle est \emph{satisfaisable}, \emph{réfutable}, \emph{tautologie} ou \emph{contradiction}.
\begin{enumerate}[label=\arabic*)]
  \item $(p\Rightarrow q)\ \lor\ (q\Rightarrow p)$.
  \item $(p \Leftrightarrow q)\ \land\ \bigl(p \Leftrightarrow (\neg q)\bigr)$.
\end{enumerate}
% (Précisez la/les ligne(s) rendant la formule vraie/fausse pour justifier la classification.)

\section*{Exercice 13 - Tautologies ? (méthode des tables de vérité)}
Dire si chacune des formules suivantes est une tautologie (justifier par table de vérité) :
\begin{enumerate}[label=\arabic*)]
  \item $p \lor \neg p$ \hfill (principe du tiers exclu)
  \item $\neg(p \land \neg p)$ \hfill (principe de non-contradiction)
  \item $(p \lor q) \Rightarrow (q \lor p)$ \hfill (commutativité de $\lor$)
  \item $p \Rightarrow (\,q \Rightarrow p\,)$ \hfill (le vrai est impliqué par tout)
  \item $\neg p \Rightarrow (\,p \Rightarrow q\,)$ \hfill (le faux implique tout)
  \item $(\,\neg p \Rightarrow p\,) \Rightarrow p$ \hfill (preuve par l'absurde)
  \item $\bigl((\neg p \Rightarrow q)\ \land\ (\neg p \Rightarrow \neg q)\bigr)\Rightarrow p$ \hfill (preuve par l'absurde)
  \item $\bigl((p \Rightarrow q)\ \land\ (q \Rightarrow r)\bigr)\Rightarrow(p \Rightarrow r)$ \hfill (transitivité de $\Rightarrow$)
\end{enumerate}

\section*{Exercice 14 - Raisonnement par l'absurde}
Démontrer, \textbf{par l'absurde} :
\begin{enumerate}[label=\arabic*.]
  \item La somme d'un rationnel et d'un irrationnel est irrationnelle, et le \textbf{produit d'un rationnel non nul} par un irrationnel est irrationnel.
  \item La racine carrée d'un nombre irrationnel positif est irrationnelle.
  \item Un rectangle a pour aire $170\,\mathrm{m}^2$. Montrer que sa longueur est $>13\,\mathrm{m}$.
  \item \textit{Si $n$ est le carré d'un entier non nul, alors $2n$ n'est pas le carré d'un entier.}
  \item \textit{$\sqrt{2}$ est irrationnel} (supposer $\sqrt{2}=\dfrac{p}{q}$ sous forme irréductible, puis discuter la parité de $p$ et $q$).
\end{enumerate}

\section*{Exercice 15 - Démonstrations par contraposée}
Démontrer, \textbf{par contraposée} :
\begin{enumerate}[label=\arabic*.]
  \item Si $n^2$ (avec $n\in\mathbb{N}$) est impair, alors $n$ est impair.
  \item Si \emph{pour tout} $\varepsilon>0$, on a $a\le \varepsilon$, alors $a\le 0$ \ (avec $a\in\mathbb{R}$).
  \item Soit $a$ un réel. Si $a^2$ n'est pas un multiple entier de $16$, alors $a/2$ n'est pas un entier pair.
\end{enumerate}

\section*{Exercice 16 - Contraposition : un cas arithmétique}
On veut montrer, pour $n\ge 2$, $n\in\mathbb{N}$, la propriété
\[
\mathbf{P}:\quad \text{Si } (n^2-1) \text{ n'est pas divisible par } 8,\ \text{alors } n \text{ est pair.}
\]
\begin{enumerate}[label=\arabic*.]
  \item Rappeler la \textbf{contraposée} de $A\Rightarrow B$ (avec $A,B$ des assertions) et \textbf{démontrer l'équivalence} $(A\Rightarrow B)\ \Leftrightarrow\ (\neg B\Rightarrow \neg A)$ à l'aide d'une table de vérité.
  \item Écrire la \textbf{contraposée} de la proposition $\mathbf{P}$.
  \item Montrer qu'un entier impair $n$ s'écrit sous la forme $n=4k+r$ avec $k\in\mathbb{N}$ et $r\in\{1,3\}$.
  \item \textbf{Prouver la contraposée} obtenue en (2) à l'aide de (3).
  \item Conclure : a-t-on bien démontré la propriété de l'énoncé ?
\end{enumerate}

\section*{Exercice 17 - Distance $<\tfrac{1}{n}$ (raisonnement par l'absurde)}
Soit $n\in\mathbb{N}$. On se donne $n+1$ réels $x_0,x_1,\dots,x_n\in[0,1]$ vérifiant
\[
0 \le x_0 \le x_1 \le \cdots \le x_n \le 1.
\]
On veut démontrer par l'absurde la propriété $\mathbf{P}$ suivante :
\[
\mathbf{P}:\ \ \text{Il existe deux de ces réels qui sont distants de moins de } \frac{1}{n}.
\]
\begin{enumerate}[label=\arabic*.]
  \item Écrire, à l'aide de \textbf{quantificateurs} et des différences $x_i - x_{i-1}$, une \textbf{formule logique} équivalente à $\mathbf{P}$.
  \item Écrire la \textbf{négation} de cette formule logique.
  \item Rédiger une \textbf{démonstration par l'absurde} de $\mathbf{P}$.
\end{enumerate}

\section*{Exercice 18 - Validité logique de raisonnements}
Déterminer, pour chacun des raisonnements suivants, s'il est \textbf{logiquement valide} (valide / non valide). 
\begin{enumerate}[label=\alph*)]
  \item \textit{Tous les élèves sont charmants.\\
  Or Édouard est charmant.\\
  Donc Édouard est un élève.}

  \item \textit{Édouard est un élève.\\
  Or tous les élèves sont charmants.\\
  Donc Édouard est charmant.}

  \item \textit{Aucun élève n'est charmant.\\
  Or Édouard n'est pas charmant.\\
  Donc Édouard est un élève.}

  \item \textit{Aucun élève n'est charmant.\\
  Or Édouard est un élève.\\
  Donc il n'est pas charmant.}

  \item \textit{La plupart des élèves s'appellent Édouard.\\
  Or tous les Édouard sont charmants.\\
  Donc certains élèves sont charmants.}

  \item \textit{Tous les élèves s'appellent Édouard.\\
  Or certains Édouard ne sont pas charmants.\\
  Donc certains élèves sont charmants.}
\end{enumerate}

\section*{Exercice 19 - Trois frères et trois crayons (déduction logique)}
Trois frères, Alfred (A), Bernard (B) et Claude (C), ont chacun un crayon d'une couleur différente parmi
\textbf{bleu}, \textbf{rouge} et \textbf{vert}. On sait que :
\begin{enumerate}[label=\arabic*.]
  \item Si le crayon d'Alfred est vert, alors le crayon de Bernard est bleu.
  \item Si le crayon d'Alfred est bleu, alors le crayon de Bernard est rouge.
  \item Si le crayon de Bernard n'est pas vert, alors le crayon de Claude est bleu.
  \item Si le crayon de Claude est rouge, alors le crayon d'Alfred est bleu.
\end{enumerate}
\textbf{Question.} Déterminer la couleur de chacun des crayons (A, B, C). \\
Y a-t-il plusieurs solutions possibles ? Justifier votre réponse.

\section*{Exercice 20 - Variante corsée : trois frères, deux attributs, un menteur}
Trois frères, Alfred (A), Bernard (B) et Claude (C), ont chacun \textbf{un stylo} et \textbf{un tee-shirt}, chacun
coloré en \{bleu (B), rouge (R), vert (V)\}. Les couleurs de \textbf{stylos} forment une permutation de \{B,R,V\}; idem pour les \textbf{tee-shirts}. 
De plus, \textbf{aucun frère n'a la même couleur pour son stylo et son tee-shirt}.

On sait que les \textbf{quatre implications sur les \underline{stylos}} ci-dessous sont vraies :
\begin{enumerate}[label=(S\arabic*)]
  \item Si le stylo d'Alfred est vert, alors le stylo de Bernard est bleu.
  \item Si le stylo d'Alfred est bleu, alors le stylo de Bernard est rouge.
  \item Si le stylo de Bernard n'est pas vert, alors le stylo de Claude est bleu.
  \item Si le stylo de Claude est rouge, alors le stylo d'Alfred est bleu.
\end{enumerate}

Chaque frère énonce ensuite une phrase \textbf{sur les \underline{tee-shirts}} (et croisements stylo/tee-shirt) :
\begin{enumerate}[label=(T\arabic*)]
  \item[\textnormal{(T1)}] (Alfred)  Mon tee-shirt n'est ni bleu, ni de la couleur du stylo de Bernard. 
  \item[\textnormal{(T2)}] (Bernard)  Le tee-shirt de Claude a la couleur du stylo d'Alfred. 
  \item[\textnormal{(T3)}] (Claude)  Le tee-shirt de Bernard a la même couleur que son propre stylo. 
\end{enumerate}

Enfin, on sait que \textbf{parmi (T1), (T2), (T3), exactement une est fausse} (un seul frère ment sur sa phrase de tee-shirt).

\medskip
\noindent\textbf{Travail demandé.}
\begin{enumerate}[label=\alph*)]
  \item \textbf{Formalisation.} Introduire des variables propositionnelles du type 
  $A^p_B$ (le \emph{stylo} d'Alfred est bleu), $B^t_R$ (le \emph{tee-shirt} de Bernard est rouge), etc., 
  et traduire formellement (S1)–(S4), (T1)–(T3), ainsi que les contraintes  permutations  et 
   stylo $\neq$ tee-shirt pour chaque frère .
  \item \textbf{Déduction.} À partir de cette formalisation, déterminer la couleur \emph{unique} de chaque stylo et de chaque tee-shirt, et identifier \emph{qui ment}.
  \item \textbf{Unicité.} Montrer qu'il n'existe \emph{aucune} autre affectation satisfaisant toutes les contraintes; préciser brièvement où portent les bifurcations éliminées.
  \item \textbf{Minimalité (piège).} Parmi (S1)–(S4), déterminer si l'une est \emph{redondante} au regard du système enrichi par (T1)–(T3) et la contrainte  un seul mensonge . Justifier.
  \item \textbf{Bonus.} Remplacer l'énoncé  exactement une (T\*) est fausse  par  exactement deux (T\*) sont fausses . 
  Discuter l'existence de solutions et, le cas échéant, leur (non-)unicité.
\end{enumerate}

\medskip
\emph{Conseils.} Vous pouvez raisonner par cas, utiliser des équivalences (contraposée, $\Rightarrow \Leftrightarrow \neg P \lor Q$), 
ou modéliser en SAT léger (tableaux de vérité restreints / propagation d'implications).

\paragraph{SAT (très bref) - à titre indicatif}
\textsc{SAT} = \textbf{Satisfiability} (satisfiabilité booléenne) : avec des \emph{variables booléennes} (Vrai/Faux) et des \emph{contraintes logiques}, on demande s'il existe une affectation qui rend \emph{toutes} les contraintes vraies.
On encode souvent en \textbf{CNF} (forme normale conjonctive) : conjonction ($\land$) de \emph{clauses}, chaque clause étant une disjonction ($\lor$) de \emph{littéraux} (variable ou sa négation).

\smallskip
\textbf{Dictionnaire utile.}
\[
P\Rightarrow Q \equiv (\neg P \lor Q)
\qquad
P\Leftrightarrow Q \equiv (\neg P \lor Q)\land(\neg Q \lor P)
\]
\[
\text{ au moins un  }x_1,\dots,x_k :\ (x_1\lor\cdots\lor x_k)
\quad
\text{ au plus un  }x_1,\dots,x_k :\ \bigwedge_{i<j}(\neg x_i\lor \neg x_j)
\]
\[
\text{ exactement un  }x_1,\dots,x_k :\ (\text{au moins un})\land(\text{au plus un})
\]

\smallskip
\textbf{Mini-exemples.}
\begin{itemize}
  \item \emph{Propagation d'unités :}  Si j'ai un badge alors la porte s'ouvre; or j'ai un badge  
  \quad $\Rightarrow$ \quad $(\neg B\lor O)\land B\ \vdash\ O$.
  \item \emph{XOR :} $P\oplus Q$ (exactement un vrai) \quad $\equiv$ \quad $(P\lor Q)\land(\neg P\lor \neg Q)$.
  \item \emph{Exactement un parmi trois :} 
  $(x_1\lor x_2\lor x_3)\land(\neg x_1\lor \neg x_2)\land(\neg x_1\lor \neg x_3)\land(\neg x_2\lor \neg x_3)$.
\end{itemize}

\noindent\emph{Remarque.} L'énigme se résout aussi \emph{sans} SAT (raisonnement par cas/implications). SAT n'est ici qu'un \emph{outil de modélisation} compact.

\section*{Exercice 21 - Choisir la bonne méthode (direct / cas / contraposée / absurde)}
Pour chaque énoncé, \textbf{indiquer une méthode adaptée} puis démontrer.
\begin{enumerate}[label=\arabic*.]
  \item Si $m,n\in\mathbb{Z}$ sont impairs, alors $m^2+n^2$ est pair.
  \item Pour tout $n\in\mathbb{Z}$, $n^3-n$ est divisible par $6$.
  \item Si $x\le y$ et $x,y\in\mathbb{R}_{\ge 0}$, alors $\sqrt{x}\le \sqrt{y}$.
  \item Si $a\mid b$ et $b\mid a$ (avec $a,b\in\mathbb{Z}\setminus\{0\}$), alors $|a|=|b|$.
\end{enumerate}

\section*{Exercice 22 - Contraposée : écrire puis prouver}
\begin{enumerate}[label=\arabic*.]
  \item  Si $ab$ est pair alors $a$ est pair ou $b$ est pair  ($a,b\in\mathbb{Z}$). \\
        \emph{(i) Écrire la contraposée. (ii) Prouver l'énoncé en passant par la contraposée.}
  \item Soit $x\in\mathbb{R}$.  Si $x^2=0$ alors $x=0$ . \\
        \emph{(i) Écrire la contraposée. (ii) Démontrer.}
  \item Soit $A\subseteq B$ deux ensembles. Montrer:  $x\notin B \Rightarrow x\notin A$ . \\
        \emph{(i) Relier à la contraposée de  $x\in A \Rightarrow x\in B$ .}
\end{enumerate}

\section*{Exercice 23 - Absurde : il ne peut pas en être autrement}
\begin{enumerate}[label=\arabic*.]
  \item Il n'existe \textbf{pas} de plus petit rationnel strictement positif.
  \item Il existe une infinité de nombres premiers.
\end{enumerate}

\section*{Exercice 24 - Récurrence (classique)}
\begin{enumerate}[label=\arabic*.]
  \item Pour tout $n\ge 1$, $1+3+5+\cdots+(2n-1)=n^2$.
  \item Pour tout $n\ge 1$, $3\mid(7^n-4^n)$.
\end{enumerate}

\section*{Exercice 25 - Récurrence forte}
\begin{enumerate}[label=\arabic*.]
  \item Tout entier $n\ge 2$ s'écrit comme produit de nombres premiers.
  \item Avec des timbres de valeurs $3$ et $5$, tout montant $m\ge 8$ peut être obtenu.
\end{enumerate}

\section*{Exercice 26 - Invariant (raisonnement "info-friendly" )}
On dispose de jetons \textbf{Rouges} (R) et \textbf{Bleus} (B). À chaque coup, on choisit deux jetons:
\[
\text{même couleur } \Rightarrow \text{ on les remplace par \textbf{un} jeton \textbf{Rouge}};\qquad
\text{couleurs différentes } \Rightarrow \text{ on les remplace par \textbf{un} jeton \textbf{Bleu}}.
\]
On part d'une configuration contenant $R_0$ jetons rouges et $B_0$ bleus.
\begin{enumerate}[label=\arabic*.]
  \item Montrer que le nombre total de jetons diminue d'une unité à chaque coup (le jeu termine).
  \item Montrer que la \textbf{parité} du nombre de jetons \emph{rouges} est \textbf{invariante}.
  \item En déduire la couleur du jeton final en fonction de la parité de $R_0$.
\end{enumerate}
\medskip
\textit{Indication.} Travailler \emph{modulo 2} sur le nombre de jetons rouges, et examiner les trois cas (RR), (BB), (RB).

\section*{Exercice 27 - Récurrence et identités classiques}
\begin{enumerate}[label=\arabic*.]
  \item Par récurrence sur $n$, montrer $\displaystyle \sum_{k=0}^n \binom{n}{k} = 2^n$.
  \item Montrer $\displaystyle \sum_{k=0}^n k\binom{n}{k} = n\,2^{\,n-1}$ (indice : dériver $(1+x)^n$).
  \item Pour $n\ge 1$, prouver $1+\frac12+\cdots+\frac{1}{2^n} < 2$ (récurrence ou majoration télescopique).
\end{enumerate}

\section*{Exercice 28 - Tiroirs et divisibilité}
\begin{enumerate}[label=\arabic*.]
  \item (Restes modulo 7) Montrer que parmi $8$ entiers, deux ont le même reste modulo $7$.
  \item (Chaîne de divisibilité) Soient $51$ entiers distincts dans $\{1,\dots,100\}$. Prouver qu'il en existe deux tels que l'un divise l'autre.
  \item (Différences multiples) Dans tout sous-ensemble de $\{1,\dots,2n\}$ à $n+1$ éléments, prouver qu'il existe deux éléments dont la différence est $\le n$.
\end{enumerate}

\section*{Exercice 29 - Preuve de terminaison et d'exactitude (algorithme d'Euclide)}
On considère l'algorithme : \quad \textbf{tant que} $b\ne 0$ \textbf{faire} $(a,b)\leftarrow (b,\ a\bmod b)$.
\begin{enumerate}[label=\arabic*.]
  \item (\emph{Invariant}) Montrer $\gcd(a,b)=\gcd(b,\ a\bmod b)$ à chaque itération.
  \item (\emph{Variant}) Justifier que la suite des secondes composantes $b$ décroît strictement et reste $\ge 0$ (terminaison).
  \item (\emph{Correction}) Conclure que l'algorithme renvoie $\gcd(a_0,b_0)$.
\end{enumerate}

%\section*{Exercice 30 - Pour aller plus loin : mini-Sudoku 4$\times$4 (option SAT)}
%On remplit une grille $4\times 4$ avec les chiffres $\{1,2,3,4\}$ de sorte que chaque \textbf{ligne}, chaque \textbf{colonne} et chaque \textbf{bloc $2\times 2$} contiennent chacun \emph{exactement une fois} chaque chiffre.
%
%\medskip
%\noindent\textbf{Grille.}
%\[
%\begin{array}{|c|c||c|c|}
%\hline
%\phantom{0} & 2 & \phantom{0} & \phantom{0} \\\hline
%\phantom{0} & \phantom{0} & 3 & \phantom{0} \\\hline\hline
%3 & \phantom{0} & \phantom{0} & 2 \\\hline
%\phantom{0} & \phantom{0} & 1 & \phantom{0} \\\hline
%\end{array}
%\]
%
%\medskip
%\noindent\textbf{Travail demandé.}
%\begin{enumerate}[label=\alph*)]
%  \item (\emph{Raisonnement pur}) Résoudre la grille par déductions logiques (lignes/colonnes/blocs).
%  \item (\emph{Modélisation}) Introduire des booléennes $X_{r,c}^{(d)}$ :  la case $(r,c)$ vaut $d$ . Écrire les contraintes “\textbf{exactement un}” :
%  \begin{itemize}
%    \item par \textbf{case} : pour chaque $(r,c)$, exactement un $d\in\{1,2,3,4\}$ est vrai ;
%    \item par \textbf{ligne} : pour chaque $r$ et chaque $d$, exactement une colonne $c$ vérifie $X_{r,c}^{(d)}$ ;
%    \item par \textbf{colonne} : pour chaque $c$ et chaque $d$, exactement une ligne $r$ vérifie $X_{r,c}^{(d)}$ ;
%    \item par \textbf{bloc} : même principe pour chaque bloc $2\times 2$ et chaque $d$.
%  \end{itemize}
%  \item (\emph{Option SAT courte}) En CNF (forme conjonctive), “exactement un parmi $x_1,\dots,x_k$” s'écrit :
%  \[
%    (x_1\lor\cdots\lor x_k)\ \land\ \bigwedge_{i<j}(\neg x_i\lor \neg x_j).
%  \]
%  Encoder deux familles de contraintes au choix (par case + par ligne, ou par case + par bloc).
%  \item (\emph{Unicité}) La solution est-elle unique ? Justifier brièvement.
%\end{enumerate}
%
%\medskip
%\noindent\textit{Remarque.} L'option SAT est \emph{facultative}. Elle illustre comment traduire un problème combinatoire en contraintes logiques compactes.

\section*{Exercice 30 - Pour aller plus loin : synthèse des méthodes de raisonnement}
On veut relier et mettre en pratique quatre techniques : \textbf{induction simple}, \textbf{induction forte},
\textbf{principe du bon ordre} (tout sous-ensemble non vide de $\mathbb{N}$ possède un plus petit élément), 
et \textbf{méthode du plus petit contre-exemple / descente infinie}.

\begin{enumerate}[label=\alph*)]
  \item (\textbf{Équivalences de principes})  
  Montrer que, sur $\mathbb{N}$, les principes suivants sont \emph{équivalents} :
  \begin{enumerate}[label=\roman*)]
    \item Induction simple : $P(0)$ et $\forall n\,(P(n)\Rightarrow P(n{+}1)) \Rightarrow \forall n\,P(n)$.
    \item Induction forte : $P(0)$ et $\forall n\,\big((\forall k\le n\,P(k))\Rightarrow P(n{+}1)\big)\Rightarrow \forall n\,P(n)$.
    \item Bon ordre : tout $S\subseteq\mathbb{N}$ non vide admet un \emph{minimum}.
  \end{enumerate}
  \emph{Indication :} (iii)$\Rightarrow$(i) par l'ensemble des contre-exemples; (i)$\Rightarrow$(ii) en  empilant  l'hérédité; (ii)$\Rightarrow$(iii) par contraposée.

  \item (\textbf{Plus petit contre-exemple})  
  Prouver par la méthode du \emph{plus petit contre-exemple} que pour tout $n\ge 2$, 
  $n$ admet un \textbf{diviseur premier} $\le \sqrt{n}$.  
  \emph{Indication :} supposer faux, prendre le plus petit $n$ sans tel diviseur, factoriser $n=ab$ avec $2\le a\le b$, 
  et utiliser la minimalité.

  \item (\textbf{Descente infinie})  
  Montrer qu'il n'existe \textbf{aucuns} entiers positifs $(x,y)$ tels que $x^2=3y^2$.  
  \emph{Indication :} raisonner par l'absurde; à partir d'une solution minimale pour $x$, construire une solution plus petite.

  \item (\textbf{Induction forte - existence/unicité})  
  \begin{enumerate}[label=\roman*)]
    \item (\emph{Existence}) Par \textbf{induction forte}, démontrer que tout entier $n\ge 2$ se factorise en produit de nombres premiers.
    \item (\emph{Unicité}) Par \textbf{plus petit contre-exemple}, montrer que cette factorisation est \emph{unique} à l'ordre près.
  \end{enumerate}

  \item (\textbf{Comparaison des preuves})  
  Choisir l'un des énoncés (b), (c) ou (d.i) et en donner \textbf{deux} démonstrations distinctes 
  parmi : induction simple, induction forte, bon ordre / plus petit contre-exemple, absurde.
  Discuter en quelques lignes les avantages de chaque stratégie.
\end{enumerate}

% ===================== ANNEXE : CORRIGÉS =====================
\clearpage
\appendix
\section{Corrigés}

\section*{Corrigé - Exercice 20 (variante corsée)}

\paragraph{1) Stylos (S1–S4) \;:\; déduction et unicité.}
On note les couleurs $B$ (bleu), $R$ (rouge), $V$ (vert). Les stylos forment une permutation de $\{B,R,V\}$.
\begin{itemize}
  \item \textbf{Cas $A^p=B$.} Par (S2) \;$A^p=B \Rightarrow B^p=R$. Alors $B^p\ne V$ donc (S3) impose $C^p=B$, ce qui violerait la permutation ($B$ déjà chez $A$). \textit{Impossible}.
  \item \textbf{Cas $A^p=V$.} Par (S1) \;$A^p=V \Rightarrow B^p=B$. Alors $B^p\ne V$ donc (S3) impose $C^p=B$, encore une duplication. \textit{Impossible}.
  \item \textbf{Donc $A^p=R$.} Alors (S4) impose $C^p\ne R$ (car $C^p=R \Rightarrow A^p=B$ serait faux). 
        Par (S3), si $B^p\ne V$ alors $C^p=B$; cela duplique $B$ si on mettait $B^p=B$. On en déduit \;$B^p=V$ et \;$C^p=B$.
\end{itemize}
\[
\boxed{\,A^p=R,\quad B^p=V,\quad C^p=B\,}\quad\text{(unique).}
\]

\paragraph{2) Tee-shirts et phrases (T1–T3) avec les contraintes du sujet.}
Contraintes tee-shirts : permutation de $\{B,R,V\}$ \textbf{et} pour chaque frère, \emph{tee-shirt $\neq$ stylo}.

\smallskip
\noindent Avec $A^p=R,\,B^p=V,\,C^p=B$ :
\begin{align*}
&\text{T1 (Alfred) : } A^t\not= B \ \text{et}\ A^t\not= B^p(=V) \ \Longrightarrow\ A^t=R,\\
&\text{T2 (Bernard) : } C^t = A^p = R,\\
&\text{T3 (Claude) : } B^t = B^p = V.
\end{align*}
Or les contraintes imposent \emph{aussi} $A^t\not= A^p(=R)$, $B^t\not= B^p(=V)$, $C^t\not= C^p(=B)$. 
Ainsi, \textbf{T1 vraie} forcerait $A^t=R$ \emph{interdit}; \textbf{T3 vraie} forcerait $B^t=V$ \emph{interdit}. 
Donc sous les contraintes données, on doit avoir \emph{au moins} T1 et T3 fausses. \\
\[
\Rightarrow\ \textbf{Il est impossible d'avoir  exactement une  des (T1–T3) fausse : \emph{énoncé insatisfaisable}.}
\]

\paragraph{3) Correction A (minimale) - remplacer  exactement une  par  exactement deux  fausses.}
On conserve toutes les contraintes (\emph{tee-shirt $\neq$ stylo} + permutation) et on suppose \emph{exactement deux} des (T1–T3) fausses.
On peut réaliser :
\[
C^t=R\ \ (\text{T2 vraie}),\qquad B^t=B,\qquad A^t=V.
\]
Vérifications : $A^t\not=A^p$, $B^t\not=B^p$, $C^t\not=C^p$ et $\{A^t,B^t,C^t\}=\{B,R,V\}$.
\[
\text{T1: }A^t=V\ \Rightarrow\ \text{ ni bleu ni vert  est \emph{fausse}};\qquad
\text{T2: }C^t=R\ \Rightarrow\ \text{vraie};\qquad
\text{T3: }B^t=B\not=V\ \Rightarrow\ \text{fausse}.
\]
On a bien \textbf{exactement deux} fausses (T1 et T3). Cette affectation est \emph{unique} sous ces contraintes.

\paragraph{4) Correction B (alternative) - garder  exactement une  fausse, mais supprimer  tee-shirt $\neq$ stylo .}
Si l'on retire la contrainte $($tee-shirt différent du stylo pour chaque frère$)$ et qu'on garde la permutation des tee-shirts, 
on peut prendre
\[
C^t=R\ (\text{T2 vraie}),\quad B^t=V\ (\text{T3 vraie}),\quad A^t=B.
\]
Alors T1 ( A$^t$ n'est ni bleu ni couleur de $B^p=V$ ) est \textbf{fausse} (car $A^t=B$) et c'est la \textbf{seule} fausse. 
Toutes les autres contraintes sont satisfaites.

\paragraph{Conclusion.}
\begin{itemize}
  \item Dans la version initiale (avec  tee-shirt $\neq$ stylo  \emph{et}  exactement une est fausse ), l'énigme est \textbf{incohérente}.
  \item Deux réparations naturelles :
  \begin{itemize}
    \item \textbf{A} :  exactement deux sont fausses  $\Rightarrow$ solution unique \;$A^t=V,\ B^t=B,\ C^t=R$.
    \item \textbf{B} : retirer  tee-shirt $\neq$ stylo  $\Rightarrow$  exactement une fausse  réalisable et unique \;$A^t=B,\ B^t=V,\ C^t=R$.
  \end{itemize}
\end{itemize}

% -------------------------------------------------------------

\section*{Corrigé - Exercice 30}

\begin{enumerate}[label=\alph*)]

\item \textbf{(Équivalences de principes)}

\emph{(iii) $\Rightarrow$ (i).} Supposons le \emph{bon ordre} vrai. Soit $S=\{n\in\mathbb{N}\mid \neg P(n)\}$ l'ensemble des contre-exemples. 
Si $S\neq\varnothing$, il admet un minimum $m$. On ne peut avoir $m=0$ puisque $P(0)$ est vraie. 
Donc $m>0$ et $P(m-1)$ est vraie; par hérédité $P(m)$ est vraie, contradiction. Ainsi $S=\varnothing$ et $\forall n\,P(n)$.

\emph{(i) $\Rightarrow$ (ii).} Posons $Q(n):\;(\forall k\le n,\ P(k))$. On a $Q(0)$ par $P(0)$. 
Si $Q(n)$ est vraie, l'hypothèse d'induction \emph{forte} donne $P(n+1)$, donc $Q(n+1)$ est vraie. 
Par \emph{induction simple} sur $Q$, on obtient $\forall n,\ Q(n)$, donc $\forall n,\ P(n)$.

\emph{(ii) $\Rightarrow$ (iii) (par contraposée).} Supposons qu'il existe $S\subseteq\mathbb{N}$ non vide \emph{sans} minimum. 
Posons $P(n):\; n\notin S$. Alors $P(0)$ est vraie (sinon $0$ serait le minimum de $S$). 
Si $\forall k\le n,\ P(k)$ et si $n+1\in S$, alors $n+1$ serait le plus petit élément de $S$, contradiction; donc $P(n+1)$. 
Par \emph{induction forte}, $\forall n,\ P(n)$; ainsi $S=\varnothing$, contradiction. Donc le \emph{bon ordre} est vrai.

\item \textbf{(Plus petit contre-exemple)}
\smallskip

\emph{Forme correcte.} Pour $n\ge 2$ \textbf{composé}, il existe un \textbf{diviseur premier} $p$ de $n$ tel que $p\le \sqrt{n}$.
(En effet, si $n$ est premier et $n>\sqrt{n}$, la version  $p\le \sqrt{n}$  est fausse.)

\emph{Preuve (plus petit contre-exemple).} Supposons qu'il existe un $n\ge 2$ composé \emph{sans} diviseur premier $\le \sqrt{n}$, 
et choisissons-le \emph{minimal}. Comme $n$ est composé, $n=ab$ avec $2\le a\le b<n$. Alors $a\le \sqrt{n}$ (sinon $ab> \sqrt{n}\cdot \sqrt{n}=n$).
Si $a$ est premier, on a trouvé $p=a\le \sqrt{n}$ (contradiction). 
Si $a$ est composé, par minimalité de $n$, $a$ admet un \emph{diviseur premier} $p\le \sqrt{a}\le \sqrt{n}$; alors $p\mid n$, contradiction. 
Donc l'assertion corrigée est démontrée.

\item \textbf{(Descente infinie)}
\smallskip

Supposons qu'il existe des entiers positifs $(x,y)$ tels que $x^2=3y^2$, et choisissons-en un avec $x$ minimal. 
Alors $3\mid x$, donc $x=3x_1$; en substituant: $9x_1^2=3y^2\Rightarrow y^2=3x_1^2$, donc $3\mid y$ et $y=3y_1$. 
On obtient $(x_1,y_1)$ \emph{plus petit} vérifiant la même équation, contradiction avec la minimalité de $x$. 
Donc aucune solution entière positive n'existe.

\item \textbf{(Induction forte - existence / unicité)}

\emph{(i) Existence (induction forte).} Pour $n=2$, c'est premier. Soit $n\ge 2$ tel que tout $m$ avec $2\le m\le n$ 
se factorise en produit de nombres premiers. Pour $n+1$, soit $n+1$ est premier, soit $n+1=ab$ avec $2\le a\le b\le n$. 
Par l'hypothèse d'induction forte, $a$ et $b$ sont produits de nombres premiers; donc $n+1$ l'est aussi.

\emph{(ii) Unicité (plus petit contre-exemple + lemme d'Euclide).} 
Supposons l'existence d'un $n\ge 2$ admettant \emph{deux} factorisations en nombres premiers distinctes et prenons le plus petit. 
Écrivons $n=p_1\cdots p_r=q_1\cdots q_s$ avec $p_1\le\cdots\le p_r$ et $q_1\le\cdots\le q_s$. 
Le \textbf{lemme d'Euclide} dit: si $p$ est premier et $p\mid ab$, alors $p\mid a$ ou $p\mid b$. 
Comme $p_1\mid n$, on a $p_1\mid q_j$ pour un certain $j$, donc $p_1=q_j$. En supprimant ce facteur commun, 
on obtient une factorisation double d'un entier $<n$, contredisant la minimalité. 
Donc la factorisation en nombres premiers est \emph{unique} à l'ordre près.

\item \textbf{(Deux preuves pour un même énoncé) - choix : partie (c)}
\smallskip

\emph{Preuve 1 (descente, ci-dessus).} Donne immédiatement l'impossibilité d'une solution entière positive.

\emph{Preuve 2 (arithmétique modulaire).} Modulo $3$, un carré est $\equiv 0$ ou $\equiv 1$. 
De $x^2=3y^2$, on déduit $x^2\equiv 0\pmod 3$, donc $3\mid x$ et $x=3x_1$. 
En remplaçant, $9x_1^2=3y^2$ donc $y^2\equiv 0\pmod 3$ et $3\mid y$. 
En itérant, on obtient une infinité de divisibilités par $3$, impossible pour des entiers non nuls. 
Donc il n'existe pas de solution.

\end{enumerate}

\end{document}
