\documentclass[aspectratio=169]{beamer}

% --------- Thème & langue ----------
\usetheme{Madrid}
\usepackage[french]{babel}
\usepackage[T1]{fontenc}
\usepackage[utf8]{inputenc}
\usefonttheme{professionalfonts}
\usepackage{lmodern}
\usepackage{mathrsfs,amsmath,amssymb,amsfonts,mathtools}
\usepackage{xcolor,graphicx}
\usepackage{tikz}
\usepackage{booktabs}
%\usepackage{enumitem}

% --------- Couleurs / styles ----------
\definecolor{fsgaBlue}{RGB}{0,91,171}
\setbeamercolor{structure}{fg=fsgaBlue}
\setbeamercolor{title}{fg=white,bg=fsgaBlue}
\setbeamercolor{frametitle}{fg=fsgaBlue}
\setbeamercolor{block title}{bg=fsgaBlue!10,fg=fsgaBlue}
\setbeamercolor{block body}{bg=fsgaBlue!3}
\setbeamertemplate{navigation symbols}{}
\setbeamertemplate{footline}[frame number]

% --------- Raccourcis ----------
\newcommand{\N}{\mathbb{N}}
\newcommand{\Z}{\mathbb{Z}}
\newcommand{\Q}{\mathbb{Q}}
\newcommand{\R}{\mathbb{R}}
\newcommand{\C}{\mathbb{C}}
\DeclareMathOperator{\Card}{Card}

% --------- Logos (placer logo.png / logo2.png à côté du .tex) ----------
\title[Ensembles, relations \& fonctions]{Ensembles, relations \& fonctions}
\subtitle{Mathématiques pour Informaticiens}
\author[Geovany B. P. Laguerre]{Geovany Batista Polo LAGUERRE \textbar{} Data Scientist}
\institute[FSGA \& UniQ]{FSGA - Université Quisqueya}
\date{\today}

\titlegraphic{%
  \vspace{-0.8cm}\hspace{0.5cm}%
  \includegraphics[height=1.1cm]{logo.png}\hfill
  \includegraphics[height=1.1cm]{logo2.png}\hspace{0.5cm}
}

% --------- Environnements ----------
\newtheorem{theoreme}{Théorème}
\newtheorem{proposition}{Proposition}
\newtheorem{corollaire}{Corollaire}
\theoremstyle{definition}
\newtheorem{definitionenv}{Définition}
\newtheorem{exemple}{Exemple}
\newtheorem{exercice}{Exercice}

\begin{document}

% ===== Page de titre =====
\begin{frame}
  \titlepage
\end{frame}

\AtBeginSection[]{
  \begin{frame}{Plan}
    \tableofcontents[currentsection]
  \end{frame}
}

% ===== Objectifs =====
\section{Objectifs}
\begin{frame}{Objectifs de la séance}
\begin{itemize}
  \item Maîtriser les notions d'\textbf{ensemble} et d'\textbf{opérations} (union, intersection, complémentaire) ;
  \item Manipuler le \textbf{produit cartésien} et les \textbf{applications} (composition, image, image réciproque) ;
  \item Caractériser \textbf{injection, surjection, bijection} et l'\textbf{inverse} d'une bijection ;
  \item Introduire les \textbf{cardinaux}, le \textbf{principe des tiroirs} et quelques dénombrements utiles ;
  \item Définir une \textbf{relation d'équivalence} et les \textbf{classes}, aperçu des \textit{partitions}.
\end{itemize}
\end{frame}

% ===== Motivation =====
\section{Motivation}
\begin{frame}{Motivation - un clin d'œil à Russell}
\begin{block}{Idée}
Le langage naïf des ensembles mène à des paradoxes (ex.: « ensemble de tous les ensembles qui ne se contiennent pas eux-mêmes »).
Objectif du cours: manipuler des \textbf{définitions précises} et des \textbf{preuves rigoureuses}, sans entrer dans l'axiomatique profonde.
\end{block}
\end{frame}

% ===== Définir un ensemble =====
\section{Définir un ensemble}
\begin{frame}{Définir un ensemble}
\begin{definitionenv}
Un \textbf{ensemble} est une collection d'objets (éléments) distincts. On note $x\in E$ si $x$ appartient à $E$.
Deux ensembles sont égaux s'ils ont les \textbf{mêmes éléments}.
\end{definitionenv}
\medskip
\textbf{Inclusion} : $A\subseteq B \iff (\forall x)\,[x\in A\Rightarrow x\in B]$.
\end{frame}

% ===== Opérations =====
\section{Opérations}
\begin{frame}{Opérations ensemblistes}
\begin{columns}[T]
\column{0.55\textwidth}
\begin{itemize}
  \item $A\cup B=\{x: x\in A\text{ ou }x\in B\}$ ;
  \item $A\cap B=\{x: x\in A\text{ et }x\in B\}$ ;
  \item $A\setminus B=\{x: x\in A\text{ et }x\notin B\}$ ;
  \item $A^{c}$ : complémentaire (dans un univers $U$ fixé).
\end{itemize}
\column{0.45\textwidth}
\begin{block}{De Morgan}
$(A\cup B)^{c}=A^{c}\cap B^{c}$,\quad
$(A\cap B)^{c}=A^{c}\cup B^{c}$.
\end{block}
\end{columns}
\end{frame}

% ===== Produit cartésien =====
\section{Produit cartésien}
\begin{frame}{Produit cartésien}
\begin{definitionenv}
$E\times F=\{(x,y):x\in E,y\in F\}$.
\end{definitionenv}
\begin{exemple}
Si $E=\{a,b\}$ et $F=\{1,2,3\}$ alors $E\times F=\{(a,1),(a,2),(a,3),(b,1),(b,2),(b,3)\}$.
\end{exemple}
\end{frame}

% ===== Applications =====
\section{Applications}
\begin{frame}{Applications (fonctions)}
\begin{definitionenv}
Une \textbf{application} $f:E\to F$ associe à chaque $x\in E$ un unique $f(x)\in F$.
\end{definitionenv}
\begin{itemize}
  \item \textbf{Composition} : $(g\circ f)(x)=g(f(x))$ ;
  \item \textbf{Image directe} : $f(A)=\{f(x):x\in A\}$ ;
  \item \textbf{Image réciproque} : $f^{-1}(B)=\{x\in E:f(x)\in B\}$ (toujours définie).
\end{itemize}
\end{frame}

% ===== Injection/Surjection/Bijection =====
\section{Injection/Surjection/Bijection}
\begin{frame}{Injection, Surjection, Bijection}
\begin{columns}[T]
\column{0.58\textwidth}
\begin{description}
  \setlength\itemsep{3pt}
  \item[Injection] $f(x)=f(y)\Rightarrow x=y$.
  \item[Surjection] $\forall y\in F,\ \exists x\in E: f(x)=y$.
  \item[Bijection] $f$ est injective et surjective.
\end{description}

\column{0.42\textwidth}
\begin{block}{Inverse}
$f$ bijective $\iff$ il existe $f^{-1}:F\to E$ avec
$f^{-1}\!\circ\! f=\mathrm{id}_E$ et $f\!\circ\! f^{-1}=\mathrm{id}_F$.
\end{block}
\end{columns}
\end{frame}

% ===== Théorème bijection-inverse =====
\section{Théorème bijection-inverse}
\begin{frame}{Théorème - Bijection $\Longleftrightarrow$ existence d'un inverse}
\begin{theoreme}
Une application $f:E\to F$ est bijective \underline{ssi} il existe $g:F\to E$ telle que $g\circ f=\mathrm{id}_E$ et $f\circ g=\mathrm{id}_F$.
\end{theoreme}
\begin{proof}[Idée de preuve]
($\Rightarrow$) Si $f$ bijective, définir $g(y)$ comme l'unique antécédent de $y$: $g=f^{-1}$.  
($\Leftarrow$) Si $g$ existe, alors $f$ est injective (car $f(x)=f(x')\Rightarrow g(f(x))=g(f(x'))\Rightarrow x=x'$) et surjective (pour tout $y$, $f(g(y))=y$).
\end{proof}
\end{frame}

% ===== Cardinaux, tiroirs =====
\section{Cardinaux, tiroirs}
\begin{frame}{Ensembles finis - cardinaux et tiroirs}
\begin{definitionenv}
Si $E$ fini, son \textbf{cardinal} $\Card(E)$ est le nombre d'éléments.
\end{definitionenv}
\begin{theoreme}[Principe des tiroirs]
Si $\Card(E)>\Card(F)$, toute application $E\to F$ \emph{n'est pas injective}.
\end{theoreme}
\begin{proof}[Preuve éclair]
Supposons injective : alors $|E|\le |F|$. Contradiction avec $|E|>|F|$.
\end{proof}
\end{frame}

% ===== Dénombrements utiles =====
\section{Dénombrements utiles}
\begin{frame}{Dénombrements utiles}
\begin{itemize}
  \item Nombre d'applications $f:E\to F$ si $\Card(E)=m$ et $\Card(F)=n$ : $n^{m}$.
  \item Nombre d'injections $E\hookrightarrow F$ ($m\le n$) : $n(n-1)\cdots(n-m+1)$.
  \item Nombre de parties de un ensemble à $n$ éléments : $2^{n}$.
\end{itemize}
\end{frame}

% ===== Binôme de Newton =====
\section{Binôme de Newton}
\begin{frame}{Coefficients binomiaux \& Binôme de Newton}
\begin{definitionenv}
$\displaystyle \binom{n}{k}=\frac{n!}{k!(n-k)!}$ pour $0\le k\le n$.
\end{definitionenv}
\begin{theoreme}[Binôme de Newton]
$\displaystyle (x+y)^{n}=\sum_{k=0}^{n}\binom{n}{k}x^{n-k}y^{k}$.
\end{theoreme}
\begin{proof}[Preuve combinatoire]
Développer $(x+y)^n$ revient à choisir, pour chaque facteur, $x$ ou $y$.  
Le terme $x^{n-k}y^{k}$ apparaît pour chaque choix de $k$ positions où l'on prend $y$ : il y en a $\binom{n}{k}$.
\end{proof}
\end{frame}

% ===== Relations d'équivalence =====
\section{Relations d'équivalence}
\begin{frame}{Relations et équivalences}
\begin{definitionenv}
Une relation $\sim$ sur $E$ est une \textbf{équivalence} si elle est \textbf{réflexive}, \textbf{symétrique} et \textbf{transitive}.
\end{definitionenv}
\begin{proposition}
Les classes d'équivalence forment une \textbf{partition} de $E$ ; réciproquement, toute partition définit une équivalence.
\end{proposition}
\end{frame}

% ===== EXERCICES corrigés =====
\section{EXERCICES corrigés}
\begin{frame}{Exercice 1 - Injection/Surjection/Bijection}
Soit $f:\R\to\R,\ f(x)=ax+b$ avec $a,b\in\R$.
\begin{enumerate}
  \item Pour quelles valeurs de $a$ la fonction est-elle injective ? surjective ? bijective ?
\end{enumerate}
\pause
\textbf{Solution.} Si $a=0$, $f$ est constante $\Rightarrow$ ni injective ni surjective.  
Si $a\ne 0$, $f$ est strictement monotone et affine $\Rightarrow$ bijective (donc injective et surjective).  
\end{frame}

\begin{frame}{Exercice 2 - Composition et image réciproque}
Soit $f:\R\to\R,\ f(x)=x^2$ et $g:\R\to\R,\ g(x)=x+1$. Soit $B=[1,4]$.
\begin{enumerate}
  \item Calculer $(f\circ g)(x)$ ; \quad
  \item $f^{-1}(B)$.
\end{enumerate}
\pause
\textbf{Solution.} $(f\circ g)(x)=(x+1)^2=x^2+2x+1$.  
$f^{-1}([1,4])=\{x\in\R:1\le x^2\le 4\}=(-\infty,-1]\cup[1,2]\cup[-2,-1]$ ; en simplifiant: $[-2,-1]\cup[1,2]$.
\end{frame}

\begin{frame}{Exercice 3 - Tiroirs}
On place $13$ objets dans $12$ tiroirs. Montrer qu'au moins un tiroir contient $\ge 2$ objets.
\pause

\textbf{Solution.} Si chaque tiroir contenait 0 ou 1 objet, on ne placerait au plus que 12 objets. Contradiction.  
Donc un tiroir contient au moins 2 objets.
\end{frame}

\begin{frame}{Exercice 4 - Dénombrement de fonctions}
Combien d'applications $f:\{1,2,3\}\to\{a,b,c,d\}$ sont surjectives ?
\pause

\textbf{Solution.} Surjection impossible car $|\mathrm{dom}|=3<4=|\mathrm{codom}|$ ; il faut au moins autant d'éléments dans le domaine que dans le codomaine pour « couvrir » toutes les valeurs.
\end{frame}

\begin{frame}{Exercice 5 - Équivalences et classes}
Dans $E=\Z$, définir $x\sim y \iff x-y$ est pair. Montrer que $\sim$ est une équivalence et décrire les classes.
\pause

\textbf{Solution.} Réflexive ($x-x=0$ pair), symétrique (si $x-y$ pair, $y-x=-(x-y)$ pair), transitive (somme de pairs).  
Classes: $[\overline{0}]=\{\text{entiers pairs}\}$ et $[\overline{1}]=\{\text{entiers impairs}\}$ (partition en deux classes).
\end{frame}

% ===== Mini-quiz (interactif en présentiel) =====
\section{Mini-quiz}
\begin{frame}{Mini-quiz (flash)}
\begin{enumerate}
  \item Vrai/Faux : l'image réciproque préserve les inclusions : $A\subseteq B \Rightarrow f^{-1}(A)\subseteq f^{-1}(B)$.
  \item Pour $|E|=m,|F|=n$, le nombre de fonctions $E\to F$ est $n^m$ (V/F ?).
  \item Si $f$ est bijective, $f^{-1}$ est-elle bijective ? (Oui/Non)
\end{enumerate}
\pause
\textbf{Réponses.} (1) Vrai ; (2) Vrai ; (3) Oui.
\end{frame}

% ===== Bonus optionnel : congruences (si tu le souhaites) =====
\section{Bonus optionnel}
\begin{frame}{Bonus (optionnel) - Congruence et classes}
Pour $n\ge 2$, la relation $a\equiv b\,[n] \iff n\mid(a-b)$ est une équivalence sur $\Z$.  
Les classes sont $[0],[1],\dots,[n-1]$ ; l'ensemble quotient $\Z/n\Z$ regroupe les résidus.
\end{frame}

% ===== Wrap-up =====
\section{Wrap-up}
\begin{frame}{Wrap-up \& transition}
\begin{itemize}
  \item Savoirs : ensembles/ops., applications, bijection \& inverse, cardinaux, tiroirs, équivalences/partitions.
  \item Savoirs-faire : \textit{raisonnements} (direct, par cas), preuves courtes (tiroirs, Newton).
\end{itemize}
\textbf{À suivre :} Relations d'ordre, relations de préférence, et premières structures combinatoires.
\end{frame}

\end{document}
